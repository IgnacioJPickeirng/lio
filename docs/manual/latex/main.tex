%\documentclass[acs,preprint,floats]{revtex4}
\documentclass[journal=jctcce,manuscript=article]{achemso}
%journal=ancac3, % for ACS Nano
%journal=acbcct, % for ACS Chem. Biol.
%journal=jacsat, % for undefined journal
%journal=jctcce, % for ACS JTCT
%manuscript=article]{achemso}
\setkeys{acs}{articletitle = true}
%\documentclass[aps,floats,prb,12pt]{revtex4} ESTE
\usepackage{graphics,dcolumn}
\usepackage{graphicx}
\usepackage{chemfig}
\usepackage[version=3]{mhchem}
\usepackage{float}
\usepackage{comment}
\usepackage{soul}
\usepackage[flushleft]{threeparttable}
\usepackage{multirow}
\renewcommand{\baselinestretch}{2}
%\bibliographystyle{unsrt}

%\begin{document}
\special{papersize=8.5in,11in}

\title{Lio}
%======================================================================

\author{NaN}
\affiliation{Departamento de Qu\'imica Inorg\'anica, Anal\'itica
y Qu\'imica F\'isica/INQUIMAE, Facultad de Ciencias Exactas
y Naturales, Universidad de Buenos Aires, Ciudad Universitaria,
Pab. II, Buenos Aires (C1428EHA) Argentina}


\begin{document}


\begin{abstract}
proyecto de manual de lio, hay q ponerle el formato adecuado
\end{abstract}

%\date{\today}
%\pacs{}
%\maketitle

\newpage

\section{Restrains}
Lio may add an extra potential term to Hamiltonian for penalty the distance between specified pairs of atoms. 

    \subsection{Implemenation}
    The implementation is a simple harmonic potential over a generalized coordinate $r$.
    
    \begin{equation}
      U=\frac{1}{2} k [r - l_0]^2  
      \label{E_restrain}
    \end{equation}
    
    $r$ may be defined as a weighted combination of distances between pairs of atoms.
    
    \begin{equation}
      r=  \sum_{i} \sum_{j>i} w_{ij} |\vec{r_i} - \vec{r_j}|
      \label{gen_coord}
    \end{equation}
    
    In this formulation the force over an atom l is:
    
    \begin{equation}
      \vec{F_l}= -k [r - l_0] \sum_{i} \sum_{j>i} w_{ij} \frac{\vec{r_{ij}}}{r_{ij}} \eta_{ijl}     
      \label{rest_force}
    \end{equation} 
    
    Where $\eta_{ijl}$ is defines as:
    
    \begin{equation*}
      \eta_{ijl} =
       \begin{cases}
         1 & \text{if $l=i$}\\
         -1 & \text{if $l=j$}\\
         0 & \text{in other case}
       \end{cases}
       \label{eta}
    \end{equation*}
    
    

    \subsection{Using Restrain}

    The number of pairs of atoms that going to be added in potential(s) in lio is defined with the variable number\_restr, and the list of distance restrains have to be added to in an extra lio.restrain file like in the following example:

    \begin{table}  [H]
      \begin{center}
      \begin{tabular}{ l c c c c c} 
         ai & aj & index &   k  &    wij   &  l0    \\
         1  &  2 &   0   &  0.1 &    1.0   & 7.86   \\
         3  &  4 &   0   &  0.1 &   -1.0   & 7.86   \\
         7  &  9 &   1   &  0.4 &    2.0   & -2.3   \\
         13 &  1 &   1   &  0.4 &    1.0   & -2.3   \\
         14 &  3 &   1   &  0.4 &   -3.0   & -2.3   \\
         14 &  2 &   2   &  0.2 &    1.0   & 0.5    \\
         8  &  5 &   3   &  0.3 &    1.0   & 3.2    \\
       \end{tabular}
       \end{center}
      \label{lio.restrain}
    \end{table}

In columns ai and aj you find the atom numbers in de QM system to be restrained, index number determine which distances contribute to a same generalized reaction coordinate. Finally k, wij and l0 are the force constant, weight of that distance in the generalized coordinate and equilibrium position in atomic units.
    
    \subsection{Examples}

    \textbf{1)In lio.in:}
    
    number\_restr = 1
    
    \textbf{in lio.restrain:}
    
    \begin{table}  [H]
      \begin{center}
      \begin{tabular}{ l c c c c c} 
         ai & aj & index &   k  &    wij   &  l0    \\
         1  &  2 &   0   &  0.1 &    1.0   & 7.86   \\
       \end{tabular}
       \end{center}
      \label{Tex1}
    \end{table}

    \textbf{Potential added to system:}
    
    \begin{equation}
      U=\frac{1}{2} 0.1 \Big{[} 1.0 |\vec{r_1} - \vec{r_2}| - 7.86\Big{]}^2  
      \label{Ex1}
    \end{equation}
    
    
    \textbf{2)In lio.in:}
    
    number\_restr = 2
    
    \textbf{in lio.restrain:}
    
    \begin{table}  [H]
      \begin{center}
      \begin{tabular}{ l c c c c c} 
         ai & aj & index &   k  &    wij   &  l0    \\
         1  &  2 &   0   &  0.1 &    1.0   & 7.86   \\
         3  &  4 &   0   &  0.1 &   -1.0   & 7.86   \\
       \end{tabular}
       \end{center}
      \label{Tex2}
    \end{table}

    \textbf{Potential added to system:}
    
    \begin{equation}
      U=\frac{1}{2} 0.1 \Big{[} 1.0 |\vec{r_1} - \vec{r_2}| - 1.0 |\vec{r_3} - \vec{r_4}| - 7.86\Big{]}^2  
      \label{Ex2}
    \end{equation}
    
    
    \textbf{3)In lio.in:}
    
    number\_restr = 4
    
    \textbf{in lio.restrain:}
    
    \begin{table}  [H]
      \begin{center}
      \begin{tabular}{ l c c c c c} 
         ai & aj & index &   k  &    wij   &  l0    \\
         1  &  2 &   0   &  0.1 &    1.0   & 7.86   \\
         3  &  4 &   0   &  0.1 &   -1.0   & 7.86   \\
         1  &  3 &   1   &  0.3 &    3.5   & -2.31   \\
         7  &  8 &   1   &  0.3 &   -2.2   & -2.31   \\
       \end{tabular}
       \end{center}
      \label{Tex3}
    \end{table}

    \textbf{Potential added to system:}
    
    \begin{equation}
      U=\frac{1}{2} 0.1 \Big{[} 1.0 |\vec{r_1} - \vec{r_2}| - 1.0 |\vec{r_3} - \vec{r_4}| - 7.86\Big{]}^2 + \frac{1}{2} 0.3 \Big{[} 3.5 |\vec{r_1} - \vec{r_3}| - 2.2 |\vec{r_7} - \vec{r_8}| +2.31\Big{]}^2 
      \label{Ex3}
    \end{equation}
    


\end{document}
